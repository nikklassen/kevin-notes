\section{Chapter S3: Space and Time}

\subsection{Special Relativity}

Einstein's Theories of Relativity:
\begin{itemize}
\item Special Relativity: usual ideas of space and time change as we approach the speed of light ($E = mc^2$)
\begin{itemize}
\item no object can travel faster than light
\item observing a object near the speed of light:
\begin{itemize}
\item time slows down
\item length contracts in direction of motion
\item mass increases
\end{itemize}
\item simultaneousness changes based on your frame of reference
\end{itemize}
\end{itemize}

Postulates of special relativity:
\begin{itemize}
\item laws of nature are the same for everyone
\item speed of light is the same for everyone
\end{itemize}

Time Dilation: \[ t_1 = t_0\sqrt{1-\bigg(\frac{v^2}{c^2}\bigg)} \]
Length Contraction: \[ l_1 = l_0\sqrt{1-\bigg(\frac{v^2}{c^2}\bigg)} \]
Mass Increase: \[ m_1 = \frac{m_0}{\sqrt{1-\bigg(\frac{v^2}{c^2}\bigg)}} \]

Since no information can be transfered faster than the speed of light objects traveling near the speed of light will perceived information at different rates since the information is moving much more slowly relative to their speed.

\subsubsection{Tests for Relativity}
Michelson-Morley experiment found evidence for the absoluteness of the speed of light in 18887.

Time dilation occurs often to subatomic particles in accelerators.

Time dilation discovered with airplanes and very precise clocks.

$E = mc^2$ verified by measurements taken of the sun.

If the speed of light were not absolute light coming from a car moving towards you would travel at 100km/hr + c and a car moving parallel to you would be see at 100 km/hr so witnessing their collision would look very odd.

\subsection{General Relativity:}

\begin{itemize}
    \item Gravity arises from distortions of spacetime
    \item Time runs slowly in gravitational fields
    \item Rapid changes in the motion of large masses can cause \emph{gravitational waves}
\end{itemize}

\textbf{The equivalence principle:} Being on Earth (gravity = 1g) is exactly equivalent to being in space accelerating at 9.8 $m/s^2$ (1g) feel the exact same.

Motion is relative. Usually how fast your perceive something is based on your velocity compared to it. The exception is light which always is seen at the same speed (called \textbf{absolute relativity})

Worldlines:
\begin{itemize}
    \item x-axis = space
    \item y-axis = time
    \item vertical line = no motion
    \item diagonal line = constant motion
\end{itemize}

What are the effects of a curved geometry for spacetime?
\begin{itemize}
    \item The shortest path between two points is a great circle
    \item Parallel lines eventually converge
    \item Angles in triangles add up to $> 180^\circ$
    \item Circumference of a circle is $< 2\pi r$
\end{itemize}

Saddle-shaped geometry
\begin{itemize}
    \item A piece of a hyperbola is the shortest distance between two points
    \item Parallel lines diverge
    \item Angles in a triangle $< 180^\circ$
    \item Circumference of a circle is $> 2\pi r$
\end{itemize}

According to the equivalence principle if you are floating freely, then your worldline is following the \textbf{straightest possible path} through spacetime.  If you feel weight, then your path is curving.


Curvature of spacetime depends on the size of the object and the mass of the object:  If we change the size of an object without changing its mass, spacetime becomes more curved near its surface (as does the strength of gravity).

In a black hole the curvature forms a ``bottomless pit'' in spacetime.  The point of no return in a black whole is the \textbf{event horizon}, where the escape velocity surpasses the speed of light.

\subsubsection{Gravity and Time}

In an accelerating spaceship, light travels more quickly from front to back than vice versa.  So we say that time is running more quickly in the front of the spaceship.  By the equivalence principle, time runs slower at lower altitudes

\subsubsection{Evidence for general relativity}

\begin{itemize}
    \item Precession of mercury
    \item Gravitational lensing: light bends around very massive objects
    \item Gravitational time dilation
    \item Gravitational waves
    \begin{itemize}
        \item movements of a massive object can produce gravitational waves
        \item we think we have observed this in the orbits of a binary neutron star system (orbit is getting smaller as energy is carried away)
    \end{itemize}
\end{itemize}
