\subsection{Assignment 12}

\subsubsection{Dark matter}

\begin{itemize}
    \item Effects the orbits of stars and gas, causing faster motion than we can account for
    \item Stellar masses only account for most of the total mass close to the center of a galaxy
    \item total mass - luminous mass - mass of hot gas = dark matter
    \item Two main options
    \begin{itemize}
        \item Ordinary - made of protons, neutrons, electrons.  Simply can't be detected
        \item Extraordinary - weakly interacting massive particles (WIMPs).  Mysterious neutrino-like particles.  This is the best bet
    \end{itemize}
    \item There isn't enough ordinary matter to explain ordinary dark matter
    \item Evidence:
    \begin{itemize}
        \item Masses measured for galaxy motions
        \item Temperature of hot gas (can be used to determine mass of galaxy)
        \item Gravitational lensing
    \end{itemize}
    \item WIMPs can't collapse because they don't radiate away their energy.  They helped protogalactic clouds collapse without collapsing themselves
    \item Dark matter lumps the universe together; accounting for expansion, galaxies are being drawn together into chains and sheet.
\end{itemize}

A rotation curve is a plot showing orbital speed versus distance form the center.
\begin{itemize}
    \item Rigid disk = proportional
    \item Solar system = decreasing exponential
    \item spiral galaxy = increases as you move away from the center then levels off
\end{itemize}

Rotation curves show us that instead of velocity decreasing as you move away from the center of a galaxy, it increases, or remains constant.  Both indicate that more mass is contained within the orbit than we would expect

$$v = \sqrt{\frac{M_r * G}{r}}$$

\begin{itemize}
    \item $M_r$ = encircled mass
    \item $r$ = radius of sphere containing the mass
\end{itemize}

Definitions

WIMPS: subatomic particles that have more mass than neutrinos but do not interact with light
Baryonic matter: Matter made from ordinary atoms
Gravitational lensing: The effect made when a massive object distorts the light coming from objects behind it

\subsubsection{Dark Energy}

Galaxies are expanding at an ever-increasing rate.  This is impossible if gravity is the only force involved as it would cause the speed of galaxies to decrease.

The energy causing this repulsion is called dark energy.

Critical density: The average density of the universe such that:
density $<$ critical density $\Rightarrow$ the universe expands at an ever decreasing rate, but never stops.
density $>$ critical density $\Rightarrow$ the universe stops expanding the collapses.

If a critical universe has an average density of one, our universe is $\approx 0.3$.  So it should be coasting.

Dark energy makes it so that instead of the rate of expansion slowing, it is actually increasing.  This also gives us the \textbf{oldest} model of the universe.

The is the age of the universe that would occur from each situation is ordered from youngest to oldest.  Youngest = recollapsing, critical, coasting, accelerating = oldest.

Dark energy fills the void needed to explain why CMB says the universe is flat.

\subsubsection{Gravitational Lensing}

The object being lensed is more widely separated when the object doing the lensing is
\begin{enumerate}
    \item more massive
    \item closer to the Earth
\end{enumerate}%

\subsubsection{Eras of the Universe}

\begin{tabular}{c|p{6cm}|c|c}
    Era name & description & ended after & temperature at end (K) \\ \hline
    Planck & all 4 forces operated as one & $10^{-43} s$ & $10^{32}$ \\
    GUT & strong electroweak forces unit as GUT force & $10^{-38} s$ & $10^{29}$ \\
    Electroweak & 3 forces operated: gravity, strong, electroweak & $10^{-10} s$ & $10^{15}$ \\
    Particle & Protons, neutrons both common & $10^{-3} s$ & $10^{12}$ \\
    Nucleosynthesis & fusion create helium nuclei & 5 minutes & $10^{9}$ \\
    Nuclei & H, He nuclei and electrons existed, but no neutral atoms & 380,000 years & 3000 \\
    Atoms & Neutral atoms existed, but not stars & & \\
    Galaxies & Stars and galaxies common & &
\end{tabular}

The Planck era was the hottest era, the galaxy era the coolest.

\subsubsection{Cosmic microwave background}

In the era of nuclei electrons were free, and photons bounced among them.  Once the age of nuclei ended the electrons were captured, and finally able to travel freely.  The temperature of the universe was about 3000 K at this point and was the peak wavelength.  Since then the wavelength has been decreasing linearly with the expansion of the universe.

Wavelength of CMB $\propto$ relative expansion of the universe.

The CMB has a perfect thermal radiation spectrum.  Since it was originally all contained in a small area, where temperature and density could equalize.

Current temperature of the CMB is approximately 2.73 K.
